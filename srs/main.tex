\documentclass{article}

\usepackage[utf8]{inputenc}
\usepackage[russian]{babel}
\usepackage[a4paper, margin=1in]{geometry}
\usepackage{graphicx}
\usepackage{amsmath}
\usepackage{wrapfig}
\usepackage{multirow}
\usepackage{mathtools}
\usepackage{pgfplots}
\usepackage{pgfplotstable}
\usepackage{setspace}
\usepackage{changepage}
\usepackage{caption}
\usepackage{csquotes}
\usepackage{hyperref}
\usepackage{listings}

\pgfplotsset{compat=1.18}
\hypersetup{
  colorlinks = true,
  linkcolor  = blue,
  filecolor  = magenta,      
  urlcolor   = darkgray,
  pdftitle   = {
    semt-report-1-srs-smirnov-shinakov
  },
}

\begin{document}

\begin{titlepage}
  \begin{center}
    \begin{spacing}{1.4}
      \large{Университет ИТМО} \\
      \large{Факультет программной инженерии и компьютерной техники} \\
    \end{spacing}    
    \vfill
    \textbf{
      \huge{Методы и средства программной инженерии.} \\
      \huge{Лабораторная работа №1.} \\
      \huge{Software Requirements Specification} \\
    }
  \end{center}
  \vfill
  \begin{center}
    \begin{tabular}{r l}
      Смирнов Виктор Игоревич  & P32131 \\
      Шиняков Артём Дмитриевич & R32372 \\
    \end{tabular}
  \end{center}
  \vfill
  \begin{center}
    \begin{large}
      2023
    \end{large}
  \end{center}
\end{titlepage}

\tableofcontents

\section{Введение}

  \subsection{Цель}
  [Specify the purpose of this SRS. The SRS should 
fully describe the external behavior of the 
application or subsystem identified. It also 
describes nonfunctional requirements, design 
constraints and other factors necessary to 
provide a complete and comprehensive description 
of the requirements for the software.]

Документ представляет из себяс процесс и требования 
к разработке сайта для международного
аэропорта Домодедово, расположенного в 
городе Москва. Требования к сайту описаны ниже.


  \subsection{Краткая сводка возможностей}
  [A brief description of the software application 
that the SRS applies to; the feature or other 
subsystem grouping; what Use-Case model(s) it 
is associated with;  and anything else that is 
affected or influenced by this document.]

Работающий сайт, предоставляющий информацию о состоянии аэропорта и рейсов, у которых одним из пунктов назначения/вылета указан Домодедово, 
новости, касающиеся данного аэроузла, возможность бронирования билетов на перелеты, план местности для ориентации пассажиров,
подробные указания пути до Домодедово, сведения о доступных поблизости местах отдыха/ночлега, юридические документы, содержащие описание 
прав и обязанностей пассажиров и посетителей, а также способы связи с представителями данной организации.
Взаимодействие с клиентом осуществляется 2 способами: 
1)Клиент связывается с компанией через контакты или социальные, указанные на сайте
2)Клиент бронирует места на рейс или в гостинице, указывая свои персональные данные
Сервер должен поддерживать роботоспособность сайта, транзакции при оплате билетов и предоставлять актуальную информацию о состоянии полетов.

  \subsection{Определения, акронимы и сокращения}
  % [This subsection should provide the definitions 
% of all terms, acronyms, and abbreviations required 
% to properly interpret the SRS.  This information 
% may be provided by reference to the project 
% Glossary.]

\begin{itemize}
    \item Посетитель или пассажир - любой человек, зашедший на сайт
    \item Личный кабинет - страница, которая отображает информацию о каждом зарегистрированном пользователе.
    \item Администратор - сотрудник аэропорта Домодедово или смежных компаний, имеющий свой личный кабинет и обладающий определенными правами,
    недоступными обычным посетителям.
    \item Интеграция - использование систем компаний-представителей платежных систем и сотрудничество с ними.
    \item Платёжная система - сервис, предоставляющий возможность перевода денег между двумя счетами.
    \item Раздел сайта - отдельная страница(возможно с ссылками на другие)
    \item База данных - упорядоченный набор структурированной информации или данных, которые обычно хранятся в электронном виде в компьютерной системе.
    \item Информационная страница - страница сайта, содержащая только текстовую и ссылочную информацию в качестве основного содержания.
    \item Титульная страница - страница, на которую попадает изначально пользователь.
    \item Согласованность данных - целостность данных, а также внутренняя непротиворечивость.
\end{itemize}

  \subsection{Ссылки}
  
\begin{itemize}
    \item Редактор Use-case диаграмм:
          \url{https://creately.com/diagram-type/use-case/}
    \item SRS шаблон:
          \url{https://docs.google.com/document/d/11aTUhjJxHqDMJGTDXDKh_8U_f2YVWKBS/edit?usp=sharing&ouid=112239579841283566048&rtpof=true&sd=true}
\end{itemize}


  \subsection{Обзор}
  
Документ разделен на части, осписанные ниже
\begin{enumerate}
      \item \textbf{Введение} \\
            Содержит общие сведения о документе. Описывает
            используемые ресурсы и определения.

      \item \textbf{Общее описание} \\
            Раздел содержит обобщенное описание проекта,
            аудитории, ограничений и подводных камней.

      \item \textbf{Требования} \\
            Раздел описывает все требования, предъявляемые
            разрабатываемой системе. Все упомянутые требования
            должны быть соблюдены.

      \item \textbf{Use-Case} \\
            Раздел описывает возможные сценарии взаимодействия
            пользователя и разрабатываемого рещения.

      \item \textbf{UML-диаграмма} \\
            Раздел состоит из UML-диаграммы, графически
            описывающей архитектуру и реализацию программной
            системы.
\end{enumerate}


\section{Общее описание}

  \subsection{Функционал продукта}
  Cайт должен давать возможность отслеживания 
всех грузовых и пассажирских рейсов вылетающих из/прибывающих в 
аэропорт Домодедово, поиска полетов по их 
параметрам и атрибутам, прочтения новостей, 
связанных с аэропортом, отслеживания состояния 
аэропорта, услуг и возможностей, которые он предоставляет. 
Также разрабатываемое решение должно предоставлять 
услугу бронирования и приобретения авиабилетов, 
отображения избранных рейсов клиента, отображение плана территории, 
возможность бронирования номеров в отелях и гостиницах города Москва, 
покупка парковочного места для авторизованных пользователей,
возможные маршруты до аэропорта и обратную 
связь с представителями Домодедово.


  \subsection{Описание пользователей}
  
\begin{enumerate}
      \item Клиенты, обладающие необходимостью в
            приобретении билета или отслеживании состояния 
            определенного рейса
            
      \item Пассажиры, желающие забронировать место
            проживания или парковочное пространство по 
            прибытии или заранее
            
      \item Администраторы и работники аэропорта
            Домодедово
            
      \item Представители компаний авиаперевозок,
            пользующиеся пространством аэропорта
\end{enumerate}


  \subsection{Факторы и зависимости}
  TODO: here we should describe what we need to launch our project

Для реализации транзакций необходимо подключиться к одной из популярных платежных систем.\\
Для отображения актуального расписания и состояния полетов необходимо соединение с базами данных авиакомпаний, 
обслуживаемых аэропортом Домодедово.

  \subsection{Ограничения}
  \begin{itemize}
      \item На разрабатываемый сайт накладываются
            ограничения изложенные в этом файле.
            
      \item Незадокументированные требования не
            учитываются при создании технического 
            решения задач.
            
      \item Ограничениями на минимальный функционал сайта
            являются требования из раздела функциональных.
            
      \item Ограниениями на используемые технологии
            являются требования из раздела нефункциональных.
\end{itemize}

% TODO: после нефункциональных посмотрим, мб допишем.


\section{Технические требования}

  \subsection{Функциональность}
  Примечание: 1sp = 24 часа.
\begin{enumerate}
      \item Система должна позволять пользователю просматривать
            табло с расписанием полетов, начальным или конечным
            пунктом которых указан аэропорт Домодедово. \\
            Приоритет: Высокий. Стабильность: Высокая. Трудоёмкость: 10 sp.

      \item Система должна содержать онлайн табло, отображающее расписание рейсов,
            должно иметь функцию поиска по номеру рейса/направлению полета. \\
            Приоритет: Высокий. Стабильность: Средняя. Трудоёмкость: 6 sp.

      \item Система должна содержать страницу с новостями, которая должна
            предусматривать функцию фильтрации по дате публикаций и по
            теме новостных записей. \\
            Приоритет: Высокий. Стабильность: Средняя. Трудоёмкость: 12 sp.

      \item Система должна содержать контакты для связи с
            представителями аэропорта Домодедово и ссылки
            на страницы в социальных сетях. \\
            Приоритет: Высокий. Стабильность: Средняя. Трудоёмкость: 2 sp.

      \item Система должна предоставлять администраторам
            доступ к базе данных клиентов сервиса
            PARKING.DME.RU. \\
            Приоритет: Высокий. Стабильность: Средняя. Трудоёмкость: 10 sp.

      \item Система должна предоставлять администратором возможность
            добавления новых новостей в ленту. \\
            Приоритет: Высокий. Стабильность: Средняя. Трудоёмкость: 10 sp.

      \item Система должна давать администраторам возможность
            создавать и удалять экстренные/временные
            оповещения об изменениях в работе аэропорта. \\
            Приоритет: Высокий. Стабильность: Высокая. Трудоёмкость: 8 sp.

      \item Система должна давать администраторам возможность
            отслеживания заполненных форм обратной связи. \\
            Приоритет: Средний. Стабильность: Низкая. Трудоёмкость: 8 sp.

      \item Система должна отображать пользователю оповещения о
            срочных/временных изменениях работы аэропорта
            Домодедово на главной странице сайта. \\
            Приоритет: Высокий. Стабильность: Высокая. Трудоёмкость: 8 sp.

      \item Система должна отображать пользователю ленту
            новостей, связанных с аэропортом Домодедово. \\
            Приоритет: Высокий. Стабильность: Средняя. Трудоёмкость: 5 sp.

      \item Система должна предоставлять пользователям возможность заполнения
            и отправки формы обратной связи. \\
            Приоритет: Низкий. Стабильность: Средняя. Трудоёмкость: 8 sp.

      \item Система должна предоставлять пользователю возможность поиска
            и отслеживания рейсов по пункту назначения,
            авиакомпании, предоставляющей услуги, или
            направлению, и дате отправления. \\
            Приоритет: Высокий. Стабильность: Высокая. Трудоёмкость: 8 sp.

      \item Система должна предоставлять пользователю возможность бронирования
            и покупки авиабилетов на рейсы, имеющие
            свободные места. \\
            Приоритет: Высокий. Стабильность: Средняя. Трудоёмкость: 30 sp.

      \item Система должна предоставлять пользователю
            регистрации личного кабинета партнера или
            пассажира. Для пассажиров долен быть
            огранизован доступ к бронировнаию парковочных
            мест на сайте PARKING.DME.RU. Для партнеров
            доступны модули для работы с cargo. \\
            Приоритет: Средний. Стабильность: Средняя. Трудоёмкость: 14 sp.

      \item Система должна предоставлять пользователю возможность
            отслеживания груза по его накладному номеру. \\
            Приоритет: Высокий. Стабильность: Средняя. Трудоёмкость: 10 sp.

      \item Система должна предоставлять пользователю возможность
            заполнения и отправки анкеты соискателя. \\
            Приоритет: Низкий. Стабильность: Низкая. Трудоёмкость: 8 sp.

      \item Система должна предоставлять пользователю возможность
            просмотра юридических документов, определяющих
            условия работы аэропорта и пердоставления
            услуг авиакомпаниями. \\
            Приоритет: Низкий. Стабильность: Низкая. Трудоёмкость: 5 sp.

      \item Система должна предоставлять пользователю возможность просмотра
            информационных страниц с описанием услуг и
            условий пользования аэропортом. Также должна
            предоставляться возможность печати отдельных
            страниц. \\
            Приоритет: Средний. Стабильность: Низкая. Трудоёмкость: 5 sp.

\end{enumerate}


  \subsection{Удобство использования}
  % [This section should include all of those requirements
%  that affect usability. For example,
% - specify the required training time for a normal
%   users and a power user to become productive at
%   particular operations
% - specify measurable task times for typical tasks
%   or base the new system’s usability requirements
%   on other systems that the users know and like
% - specify requirement to conform to common usability
%   standards, such as IBM’s CUA standards Microsoft’s 
%   GUI standards]

\begin{enumerate}
      \item Сайт должен быть разделен на 2 раздела:
            \begin{enumerate}
                  \item Для путешествий
                  \item Для грузовых перевозок
            \end{enumerate}

      \item Титульная страница раздела для путешествий
            должна содержать ленту последних новостей,
            онлайн-табло с рейсами, поиск рейсов по фильтрам,
            раздел "Популярное", содержащий ссылки на
            страницы с описанием предоставляемых услуг и
            раздел "Как добраться", показывающий все
            возможные варианты пути до аэропорта.

      \item Все страницы кроме титульной должны подчиняться
            установленному далее макету страницы: \\
            Макет состоит из 4 частей:

            \begin{enumerate}
                  \item Шапка, содержащая телефон горячей линии
                        аэропорта, возврат на главную страницу и все
                        доступные языки, на которых возможно
                        просматривать сайт. Также в шапке располагается
                        интсрумент для переключения между разделами
                        сайта.

                  \item Часть со всеми разделами информационных
                        страниц, ссылками на социальные сети аэропорта
                        Домодедово и поисковой строкой.

                  \item Главная часть, отображающая информацию
                        выбранного раздела, а также поиск рейсов и разделы
                        "Популярное" и "Как добраться".

                  \item Футер, содержащий ссылкы на форму обратной
                        связи, разделы партнёрских возможностей и
                        юридическую документацию.
            \end{enumerate}

\end{enumerate}


  \subsection{Надежность}
  
Поскольку обсуждаемый сайт является лицом 
аэропорта и им потенциально будет пользоваться
каждый клиент аэропорта, то недопустим
длительный отказ в обслуживании, ведь
в голове клиента он будет напрямую ассоциирован
с некоторыми проблемами работающего аэропорта.
Клиент может подумать, что безопасность перелетов
снижена, ведь как ее можно гарантировать, когда
какая-либо часть системы сломана. Это в свою очередь
может повлечь отказ от билетов и вследствие 
убытков для аэрокомпаний.

\subsubsection{Доступность}

Устанавливается доступность 99.999\% времени,
что соответствует следующим значениям максимальной 
длительности недоступности сервиса: 
\begin{itemize}
  \item Ежедневно:   0.86 секунд
  \item Еженедельно: 6 секунд
  \item Ежемесячно:  26 секунд
  \item Ежегодно:    5m 13s
\end{itemize}

Поскольку аэропорт работает круглосуточно, то 
и вебсайт должен быть доступен 24/7.

Доступ для технического обслуживания должен
предоставляться в любое время для возможности
незамедлительного решения каких-либо проблем,
однако техническое обслуживание не должно 
нарушать установленные требования к доступности.

При пониженной производительности необходимо 
обслуживать пользователей в порядке очереди
с приоритетами, работающими со следующем 
ранжированием:
\begin{enumerate}
  \item Работники аэропорта
  \item VIP-клиенты, при возможности 
        подтвердить их статус
  \item Клиенты, подключившиеся через 
        локальную сеть аэропорта (бесплатный wifi)
  \item Клиенты с рейсом в ближайшие 5 часов, 
        при возможности подтвердить их личность
  \item Остальные клиенты
\end{enumerate}

Также возможно ограничение предоставления каких-либо
дополнительных услуг и концентрация ресурсов на 
отображении расписания полетов (или другого).

Среднее время безотказной работы в день: 24 часа.

Среднее время на устранение проблем: 15 минут.
В случае возникновения критического сбоя в системе
необходимо возобновить его работу в течении 20 минут,
например, откатом на последний релизов на 
предыдущую версию.

\subsubsection{Актуальность данных}

\subsubsection{Ограничение ошибок}


[Requirements for reliability of the system should
 be specified here. Some suggestions follow:

- Availability—specify the percentage of time 
  available ( xx.xx\%), hours of use, maintenance 
  access, degraded mode operations, etc.

- Mean Time Between Failures (MTBF) — this is 
  usually specified in hours, but it could also 
  be specified in terms of days, months or years.

- Mean Time To Repair (MTTR)—how long is the 
  system allowed to be out of operation after it 
  has failed?

- Accuracy—specify precision (resolution) and 
  accuracy (by some known standard) that is required
  in the system’s output.

- Maximum Bugs or Defect Rate—usually expressed 
  in terms of bugs per thousand of lines of code 
  (bugs/KLOC) or bugs per function-point
  (bugs/function-point).
  
- Bugs or Defect Rate—categorized in terms of 
  minor, significant, and critical bugs: the 
  requirement(s) must define what is meant by 
  a “critical” bug; for example, complete loss 
  of data or a complete inability to use certain 
  parts of the system’s functionality.]


  \subsection{Производительность}
  [The system’s performance characteristics should 
be outlined in this section. Include specific 
response times. Where applicable, reference 
related Use Cases by name.

- response time for a transaction (average, maximum)

- throughput, for example, transactions per second

- capacity, for example, the number of customers or 
  transactions the system can accommodate

- degradation modes (what is the acceptable mode 
  of operation when the system has been degraded 
  in some manner)

- resource utilization, such as memory, disk, 
  communications, etc.
  

  \subsection{Поддерживаемость}
  % [This section indicates any requirements that 
% will enhance the supportability or maintainability
% of the system being built, including coding standards, 
% naming conventions, class libraries, maintenance access, 
% maintenance utilities.]

Каждый микросервис должен иметь следующую документацию:
\begin{enumerate}
    \item Описание доменной модели
    \item Описание таблиц в БД
    \item Open API спецификацию и Swagger страницу 
          всегда доступную в интранете
    \item Интро для нового сотрудника
    \item Базу данных всех тикетов за всю жизнь
    \item Документы с каждого митинга, в которых описаны 
          все принятые решения и указаны аргументы 
          за и против. Записи всех митингов, особенно 
          техтолков.
    \item Полная история работы системы контроля версий
          (например, чтобы найти автора того или иного класса).
\end{enumerate}

В каждом проекте должно быть настроено:
\begin{enumerate}
    \item При доступности технологии, автоматическая 
          генерирация API интерфейсов и моделей с помощью
          Open API Codegen, а также клиентов к ним.
    \item Линтеры и форматтеры, настроенные самым беспощадным способом.
    \item Автоматические тесты.
\end{enumerate}

Должна быть возможность выдачи админам прав к ограниченному 
просмотру содержимого базы данных на минимальный органиченный 
промежуток времени.

Во всех сервисаъ метрики собираются Прометеусом, 
отображаются в графане.



\section{actions}

  \subsection{Прецеденты}
  \begin{table}[h]
    \begin{tabular}{|c|c|}
    \hline
    Прецедент 1: Поиск рейса          & Описание                                                                                              \\ \hline
    Краткое описание                  & Клиент ищет рейс по доступной информации                                                              \\ \hline
    Актёры:                           & Посетитель                                                                                            \\ \hline
    Вспомогательные лица:             & -                                                                                                     \\ \hline
    Предусловия                       & Пользователь имеет данные необходимые для поиска информации о рейсе                                   \\ \hline
    Основное действие                 & -Пользователь вводит данные в поля фильтра рейса(таблицы)\\ &-Сайт выводит рейсы по результатам запроса \\ \hline
    Другие варианты развития действия & -Сайт выводит информацию об отсутствии рейса с такими данными                                         \\ \hline
    \end{tabular}
\end{table}

\begin{table}[h]
    \begin{tabular}{|c|c|}
    \hline
    Прецедент 2: Просмотр пользователем\\ информационных страниц          & Описание                                                                                              \\ \hline
    Краткое описание                  & Клиент просматривает тексты на информационных страницах                                                            \\ \hline
    Актёры:                           & Посетитель                                                                                            \\ \hline
    Вспомогательные лица:             & -                                                                                                     \\ \hline
    Предусловия                       & -                                   \\ \hline
    Основное действие                 & -Пользователь переходит на одну из\\ информационных страниц\\ &-Страница отображается пользователю \\ \hline
    Другие варианты развития действия & -                                        \\ \hline
    \end{tabular}
\end{table}

\begin{table}[h]
    \begin{tabular}{|c|c|}
    \hline
    Прецедент 3: Заполнение пользователем\\ анкеты обратной связи          & Описание                                                                                              \\ \hline
    Краткое описание                  & Клиент отправляет форму обратной связи                                                            \\ \hline
    Актёры:                           & Посетитель                                                                                            \\ \hline
    Вспомогательные лица:             & Администратор                                                                                                     \\ \hline
    Предусловия                       & Пользователь имеет повод для обращения к представительству аэропорта                                   \\ \hline
    Основное действие                 & -Пользователь вводит данные в форму и отправляет\\ &-Форма приходит администратору\\ &-Администратор принимает данные из формы \\ \hline
    Другие варианты развития действия & -Форма не отправляется/не доходит до администратора                                      \\ \hline
    \end{tabular}
\end{table}





\end{document}