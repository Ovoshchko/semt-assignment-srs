
Каждый микросервис должен иметь следующую документацию:
\begin{enumerate}
      \item Описание доменной модели
      \item Описание таблиц в БД
      \item Open API спецификацию и Swagger страницу
            всегда доступную в интранете
      \item Интро для нового сотрудника
      \item Базу данных всех тикетов за всю жизнь
      \item Документы с каждого митинга, в которых описаны
            все принятые решения и указаны аргументы
            за и против. Записи всех митингов, особенно
            техтолков.
      \item Полная история работы системы контроля версий
            (например, чтобы найти автора того или иного класса).
      \item Главная страница сервиса со всеми ссылками
\end{enumerate}

Работники обсуждать в ЛС только личные вопросы, все рабочие
не очень важные выносятся на обсуждение в рабочий чат,
а в основном общаться нужно только в комментариях к тикетам,
для сохранения истории.

В каждом проекте должно быть настроено:
\begin{enumerate}
      \item При доступности технологии, автоматическая
            генерирация API интерфейсов и моделей с помощью
            Open API Codegen, а также клиентов к ним.
      \item Линтеры и форматтеры, настроенные самым беспощадным способом.
      \item Автоматические тесты.
\end{enumerate}

Должна быть возможность выдачи админам прав к ограниченному
просмотру содержимого базы данных на минимальный органиченный
промежуток времени.

Во всех сервисах метрики собираются Прометеусом,
отображаются в графане.

Должно быть настроено 2 среды: тестовая и продовая.
