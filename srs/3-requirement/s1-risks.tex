
\begin{enumerate}
      \item \textbf{Прекращение финансирования} \\
            Государство из-за непростой экономической обстановки может 
            прекратить финансировать наш проект и заморозить процесс разработки. \\
            Тип риска: Ресурсный \\
            Вероятность: Низкая. \\
            Опасность: Очень высокая. \\
            Последствия: Проект не сдается в срок, программисты увольняются и уходят 
            к другим работодателям. \\
            Смягчение: Договориться с авиакомпаниями-резидентами о страховке и поддержке.

      \item \textbf{Некомпетентность сотрудников} \\
            Сотрудники, которых мы наймем могут оказаться недостаточно 
            квалифицированными для выполнения поставленных задач. \\
            Тип риска: Технический \\
            Вероятность: Средняя. \\
            Опасность: Высокая. \\
            Последствия: Разработка проекта замедлится, потеря денег на ЗП
            сотрудникам, формирование новых команд в процессе разработки. \\
            Смягчение: Тщательно подходить к отбору кандидатов, обучать 
            специалистов, принудить писать документацию для возможной менее
            болезненной замены программистов, выбрать популярные технологии
            для реализации проекта.

      \item \textbf{Потеря связи с сервисами, предоставляющими расписание рейсов.} \\
            Сервисы, предсотавляющие расписание рейсов могут отказать
            в обслуживании нашим сервисам из-за внутреннего сбоя,
            либо мы потеряем связь с ними из-за сетевых проблем \\
            Тип риска: Технический \\
            Вероятность: Низкая. \\
            Опасность: Средняя. \\
            Смягчение: Отображать на табло последнее состояние расписания, но с
            отметкой, что данные не актуальны и были загружены тогда-то.

      \item \textbf{Замена злоумышлиниками ссылок и контактов в случае взлома} \\
            Злоумышлиники в результате успешной хакерской атаки получат привелигерованный
            доступ к системе и смогут разместить вредоносные ссылки и подменить контакты
            на сайте, что поставит наших клиентов под угрозу. \\
            Тип риска: Форс-мажор \\
            Вероятность: Низкая. \\
            Опасность: Высокая. \\
            Последствия: Утечка персональных данных, потеря доверия пользователей. \\
            Смягчение: Вставлять контактную информацию в верстку прямо на этапе
            сборки проекта (захардкодить), оповестить пользователей о взломе,
            сделать веб-сайт недоступным, добавить проверку того, что ссылки содержат 
            наш домен.

      \item \textbf{Размещение фейковой информации в новостях сотрудником-злоумышлиником} \\
            Злоумышленник будучи администратором веб-сайта сможет разместить фейковую новость,
            наши клиенты будут в опасности.
            Тип риска: Форс-мажор \\
            Вероятность: Средняя. \\
            Опасность: Высокая. \\
            Последствия: Хакерские атаки на посетителей сайта, потеря репутации. \\
            Смягчение: Ввести контроль - обязательное ревью перед публикацией записи. \\

      \item \textbf{Отказ ДЦ} \\
            Датацентр, на котором развернуто наше приложение может отказать, например,
            из-за отключения электроэнергии или ошибки персонала. Из-за этого наш облачный
            сервис будет недоступен. \\
            Тип риска: Форс-мажор \\
            Вероятность: Средняя. \\
            Опасность: Очень-очень высокая. \\
            Последствия: Сервис не доступен. \\
            Смягчение: поднимать сразу несколько stateless сервисов,
            балансировать нагрузку, реплецировать базу данных и кеши,
            разворачивать инфраструктуру как минимум в 2 разных ДЦ.

      \item \textbf{Отказ платежной системы} \\
            Одна из платежных систем может отказать в обслуживании из-за 
            внутренних проблем. \\
            Тип риска: Технический \\
            Вероятность: Крайне низкая. \\
            Опасность: Высокая. \\
            Последствия: Невозможность купить билеты и забронировать 
            парковочное место онлайн на нашем сайте. \\
            Смягчение: Предупреждать пользователей о недоступности платежа в 
            данный момент, предложить воспользоваться другой ПС.
      
      \item \textbf{Спам в анкетах соискателя} \\
            Злоумышлиники могут спамить анкетами соискателя, тем самым
            не давая администраторам разбирать поступающие от обычных 
            пользователей сообщения. \\
            Тип риска: Форс-мажор \\
            Вероятность: Высокая. \\
            Опасность: Низкая. \\
            Последствия: Снижение эффективности работы администраторов. \\
            Смягчение: Добавить антиспам защиту.

      \item \textbf{Перебор номеров карт пользователей и прочее мошенничество} \\
            Злоумышлиники могут перебирать карты пользователей с целью покупуки
            билетов за их счет. \\
            Тип риска: Форс-мажор \\
            Вероятность: Высокая. \\
            Опасность: Высокая. \\
            Последствия: Кража денег со счетов пользователей. \\
            Смягчение: Добавление анти-фрод функциональности и способа возврата средств,
            добавление антиробота. 
      
      \item \textbf{Взлом аккаунтов клиентов} \\
            Злоумышленники могут взломать аккаунты клиентов. \\
            Тип риска: Форс-мажор \\
            Вероятность: Высокая. \\
            Опасность: Высокая. \\
            Последствия: Утечка персональных данных, хакерские атаки на пользователей. \\
            Смягчение: Просить пользователей выбирать сложные пароли, требовать 
            стойкие пароли при регистрации.

      \item \textbf{Сбой сервиса из-за ошибки программиста} \\
            Программист может случайно положить какой-нибудь сервис.
            Тип риска: Технический \\
            Вероятность: Очень высокая. \\
            Опасность: Очень высокая. \\
            Последствия: Неопределенное поведение системы. \\
            Смягчение: Строгие требования к разработке: 
            юнит-тесты, код-ревью, тестовая среда, выкатка в прод с апрувом,
            выкатка в прод в определенное время, настроенные алерты и прочие 
            инструменты наблюдаемости, возможность быстрого отката,
            возможность канареечных релизов.

      \item \textbf{DDoS атака} \\
            Злоумышленники могут устроить DDoS атаку на нашу инфраструктуру и 
            вызвать отказ в обслуживании.
            Тип риска: Форс-мажор \\
            Вероятность: Средняя. \\
            Опасность: Высокая. \\
            Последствия: Недоступность сервиса в течении атаки. \\
            Смягчение: Добавить балансировщик нагрузки, антиробот, 
            обеспечить наилучшую производительность системы.

\end{enumerate}