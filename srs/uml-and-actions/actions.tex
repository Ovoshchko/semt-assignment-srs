\subsubsection{Поиск рейса}
\textbf{Пользователь} ищет рейс по доступной информации. 
В данном действии принимает участие \textbf{Посетитель}.
Однако для инициации данного события \textbf{Пользователь}
должен иметь данные, необходимые для поиска информации о 
рейсе. \textbf{Пользователь} вводит данные в поля фильтра 
рейса(таблицы). Сайт выводит рейсы по результатам запроса
Либо же \textbf{Сайт} выводит информацию об отсутствии 
рейса с такими данными.

\subsubsection{Просмотр пользователем}
\textbf{Пользователь} просматривает тексты на информационных 
страницах. Пользователь переходит на одну из информационных 
страниц. Страница отображается пользователю.

\subsubsection{Заполнение формы}
\textbf{Пользователь} отправляет форму обратной связи или 
анкеты соискателя. Участвуют в данном действии 
\textbf{Посетитель} и \textbf{Администратор}.
Пользователь имеет повод для обращения к представительству 
аэропорта. Пользователь вводит данные в форму и отправляет.
Данные формы доступны администратору. \textbf{Администратор}
принимает данные из формы. Форма не отправляется/не 
доходит до администратора.

\subsubsection{Пользователь бронирует билеты}
Клиент покупает или бронирует билеты на рейс. Участники:
\textbf{Клиент}, \textbf{Публичный}, \textbf{Биллинг}, 
\textbf{Платежная система}, \textbf{Почтовый}. Пользователь
должен быть зарегестрирован. Действия:
\begin{enumerate}
    \item \textbf{Клиент} выбирает рейс.
    \item \textbf{Клиент} заполняет форму и отправляет ее 
          в \textbf{Публичный}.
    \item \textbf{Публичный} бронирует билет, используюя 
          сервис \textbf{Билетер}.
    \item \textbf{Публичный} просит у \textbf{Биллинга} 
          форму для ввода реквизитов для оплаты.
    \item \textbf{Биллинг} регистрирует факт запросы формы 
          для оплаты и просит \textbf{Платежную систему} 
          дать форму, после чего пересылает ее
          \textbf{Публичному}, а тот \textbf{Клиенту}.
    \item \textbf{Клиент} вводит реквизиты карты и нажимает 
          отправить, данные отправляются в.
          \textbf{Платежную систему}
    \item \textbf{Клиент} просит предоставить услугу у 
          \textbf{Публичного}.
    \item \textbf{Публичный} справшивает у \textbf{Биллинга}, 
          как там с оплатой дела.
    \item \textbf{Биллинг} справшивает у \textbf{Платежной системы}, 
          как там с оплатой дела.
    \item Если все ОК, то 
    \begin{enumerate}
        \item \textbf{Биллинг} закрывает транзакцию, которую 
              открыл при регистрации запроса на платеж, 
              если фейл, то фейлит транзакцию, если ХЗ, 
              то отвечает ХЗ.
        \item Если все ОК \textbf{Публичный} просит 
              \textbf{Билетера} выдать билет.
        \item \textbf{Билетер} отмечает забронированный 
              билет как купленный и просит \textbf{Почтового}, 
              отправить билет на почту \textbf{Пользователя}.
    \end{enumerate}
    \item Если ошибка с оплатой, то
    \begin{enumerate}
        \item \textbf{Публичный} просит \textbf{Билетера} 
              снять бронь, и отвечает \textbf{Клиенту}, 
              что тот проиграл
    \end{enumerate}
    \item Если ХЗ, что там с оплатой, то
    \begin{enumerate}
        \item \textbf{Публичный} просит \textbf{Билетера} 
              снять бронь, и отвечает \textbf{Клиенту}, 
              что тот проиграл
    \end{enumerate}
    \item ХЗ, как-то ждем, что-то делаем асинхронно, 
          успокаиваем пользователя
\end{enumerate}

\subsubsection{Пользователь бронирует парковочное место}
Клиент бронирует парковочное место. В этом действии
принимает участие \textbf{Пользователем}. Для начала
нужно, чтобы \textbf{Пользователь} зарегистрирован на 
сайте. \textbf{Пользователь} выбирает парковочное место. 
\textbf{Пользователь} заполняет данные. \textbf{Пользователь} 
одобряет транзакцию. Либо форма не отправляется/не доходит
до администратора.

\subsubsection{Пользователь регистрируется на сайте}
Клиент регистрируется на сайте (грузовом или пассажирском).
В этом действии участвует \textbf{Посетитель}. 
\textbf{Пользователь} вводит почту и получает пароль, 
\textbf{Пользователь} получает доступ к кабинету.
Либо Почта не дествительна аккаунт не регистрируется.

\subsubsection{Пользователь видит оповещение}
Клиент читает оповещение об изменении в работе аэропорта.
\textbf{Посетитель}, \textbf{Администратор}. 
\textbf{Администратор} публикует оповещение. 
\textbf{Пользователь} видит его на главной странице.
Администратор удаляет оповещение. Пользователю не 
отображается оповещение. Либо Почта не дествительна
аккаунт не регистрируется.

\subsubsection{Пользователь читает новости}
Клиент читает оповещение об изменении в работе аэропорта.
В данном событии участвует \textbf{Пользователь}.
Он перешел на новостную страницу, переходит по ссылке на 
новость и читает новость. Либо новость не найдена и 
выводится ошибка.

\subsubsection{Пользователь пользуется картой сайта}
Клиент пользуется картой сайта. Актеры: Пользователь.
\textbf{Пользователь} перешел на страницу с картой сайта,
переходит по ссылке на страницу, попадает на нужную страницу.

\subsubsection{Пользователь печатает информационную страницу}
Пользователь запрашивает страницу на печать. Актеры: 
\textbf{Посетитель}. Прежде всего \textbf{Пользователь} 
должен находится на нужной странице. Пользователь нажимает 
кнопку "печать", cайт создает копию страницы и печатает 
страницу.
