
Поскольку обсуждаемый сайт является лицом 
аэропорта и им потенциально будет пользоваться
каждый клиент аэропорта, то недопустим
длительный отказ в обслуживании, ведь
в голове клиента он будет напрямую ассоциирован
с некоторыми проблемами работающего аэропорта.
Клиент может подумать, что безопасность перелетов
снижена, ведь как ее можно гарантировать, когда
какая-либо часть системы сломана. Это в свою очередь
может повлечь отказ от билетов и вследствие 
убытков для аэрокомпаний.

\subsubsection{Доступность}

Устанавливается доступность 99.999\% времени,
что соответствует следующим значениям максимальной 
длительности недоступности сервиса: 
\begin{itemize}
  \item Ежедневно:   0.86 секунд
  \item Еженедельно: 6 секунд
  \item Ежемесячно:  26 секунд
  \item Ежегодно:    5m 13s
\end{itemize}

Поскольку аэропорт работает круглосуточно, то 
и вебсайт должен быть доступен 24/7.

Доступ для технического обслуживания должен
предоставляться в любое время для возможности
незамедлительного решения каких-либо проблем,
однако техническое обслуживание не должно 
нарушать установленные требования к доступности.

При пониженной производительности необходимо 
обслуживать пользователей в порядке очереди
с приоритетами, работающими со следующем 
ранжированием:
\begin{enumerate}
  \item Работники аэропорта
  \item VIP-клиенты, при возможности 
        подтвердить их статус
  \item Клиенты, подключившиеся через 
        локальную сеть аэропорта (бесплатный wifi)
  \item Клиенты с рейсом в ближайшие 5 часов, 
        при возможности подтвердить их личность
  \item Остальные клиенты
\end{enumerate}

Также возможно ограничение предоставления каких-либо
дополнительных услуг и концентрация ресурсов на 
отображении расписания полетов (или другого).

Среднее время безотказной работы в день: 24 часа.

Среднее время на устранение проблем: 15 минут.
В случае возникновения критического сбоя в системе
необходимо возобновить его работу в течении 20 минут,
например, откатом на последний релизов на 
предыдущую версию.

\subsubsection{Актуальность данных}

Поскольку невозможно одновременная максимизация
показателей согласованности данных и доступности 
сервиса, допустим некоторый уровень несогласованности,
ведь доступность в нашем случае намного важнее,
так как клиент может стерпеть слегка запаздаемые
данные, так как знание точного расписания вплоть до
минут никак не повлияет на его жизнь.

Поэтому данные в разных разделах сайта должны 
быть актульны со следующей точностью:
\begin{itemize}
  \item Распиание рейсов - 5 минут.
  \item Новости аэропорта - от 10 минут
        до - 6 часов в зависимости от
        важности новости.
\end{itemize}

\subsubsection{Уровень ошибок}

В данном параграфе определены значения терминов
"Грубая ошибка", "Незначительная ошибка".

Грубыми ошибками являются:
\begin{itemize}
  \item Отказ в обслуживании 
        порядочному пользователю.
  \item Нарушение уровня актуальности данных, 
        установленного выше.
  \item Наличие уязвимости.
  \item Потеря данных пользоватей.
  \item Потеря данных о полетах 
        в радиусе одной недели.
\end{itemize}

Незначительными ошибками являются:
\begin{itemize}
  \item TODO
\end{itemize}


% [Requirements for reliability of the system should
%  be specified here. Some suggestions follow:
% 
% - Availability—specify the percentage of time 
%   available ( xx.xx\%), hours of use, maintenance 
%   access, degraded mode operations, etc.
% 
% - Mean Time Between Failures (MTBF) — this is 
%   usually specified in hours, but it could also 
%   be specified in terms of days, months or years.
% 
% - Mean Time To Repair (MTTR)—how long is the 
%   system allowed to be out of operation after it 
%   has failed?
% 
% - Accuracy—specify precision (resolution) and 
%   accuracy (by some known standard) that is required
%   in the system’s output.
% 
% - Maximum Bugs or Defect Rate—usually expressed 
%   in terms of bugs per thousand of lines of code 
%   (bugs/KLOC) or bugs per function-point
%   (bugs/function-point).
%   
% - Bugs or Defect Rate—categorized in terms of 
%   minor, significant, and critical bugs: the 
%   requirement(s) must define what is meant by 
%   a “critical” bug; for example, complete loss 
%   of data or a complete inability to use certain 
%   parts of the system’s functionality.]
